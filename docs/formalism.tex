\documentclass[a4paper]{article}

\usepackage{amsmath}
\usepackage{amssymb}
\usepackage{parskip}
\usepackage{xifthen}
\usepackage{braket}

\newcommand{\decay}[2]{\ensuremath{#1 \to #2}}
\newcommand{\fourvec}[4]{\ensuremath{\left(#1; #2, #3, #4\right)}}
\newcommand{\spinor}[4]{\begin{pmatrix}#1\\#2\\#3\\#4\end{pmatrix}}
\newcommand{\cg}[6]{(#1, #2;\, #3, #4\, |\, #5, \ifthenelse{\isempty{#6}}{#2{+}#4}{#6})}

\begin{document}

Since we are concerned with two-body decay, all states will either be
at rest or have momentum back-to-back (chosen to lie in the $\hat{z}$
direction).

Unit-spin wave functions:
\begin{align}
  \phi^\mu_\pm & = \frac{1}{\sqrt{2}}\fourvec{0}{\mp1}{i}{0} \\
  \phi^\mu_0   & = \gamma\fourvec{\beta}{0}{0}{1}
\end{align}

Half-spin wave functions:
\begin{align}
  \psi^\alpha_+ & = \sqrt\frac{\gamma+1}{2}\spinor{1}{0}{\frac{\gamma\beta}{\gamma+1}}{0}\\
  \psi^\alpha_- & = \sqrt\frac{\gamma+1}{2}\spinor{0}{1}{0}{\frac{\gamma\beta}{\gamma+1}}
\end{align}

From the unit-spin wave functions, we can build a wave function of
arbitrary integer spin $J$:
\begin{equation}
  \Phi^{\mu_1\mu_2\dots\mu_J}_{M} = \sum_{\vec{m}} \prod_{i=1}^{J} \cg{i-1}{m_{0:i{-}1}}{1}{m_i}{i}{m_{0:i}}\, \phi^{\mu_i}_{m_i},
\end{equation}
where $m_{0:j} \equiv m_0 + \dots + m_j$ and $m_0 \equiv 0$; the sum is over $\vec m$ such that $m_{0:J} = M$.

The product of Clebsch-Gordan coefficients is independent of the order
of the couplings---it is only dependent on the numbers of each of spin
projection involved: $N_+$, $N_0$, and $N_-$.

Using the relations:
\begin{eqnarray}
  \cg{j}{m}{1}{\pm1}{j{+}1}{m{\pm}1} & = & \left[\frac{(j\pm m+1)(j\pm m+2)}{(2j+1)(2j+2)}\right]^{\frac{1}{2}}\\
  \cg{j}{m}{1}{0}{j{+}1}{m} & = & \left[\frac{(j-m+1)(j+m+1)}{(2j+1)(j+1)}\right]^{\frac{1}{2}}
\end{eqnarray}
and ordering the wave functions such that the $N_+$ $\phi_+$ are
first, then the $N_0$ $\phi_0$, and finally the $N_-$ $\phi_-$, we get:
\begin{eqnarray}
  C & \equiv & \prod_{i=1}^{J} \cg{i-1}{m_{0:i{-}1}}{1}{m_i}{i}{m_{0:i}} \\
  C^2 & = & \frac{(J+M)!(J-M)!}{(2J)!}2^{N_0/2}
\end{eqnarray}
So:
\begin{equation}
  \Phi^{\mu_1\mu_2\dots\mu_J}_{M} = \left[\frac{(J+M)!(J-M)!}{(2J)!}\right]^{\frac{1}{2}} \sum_{\vec{m}}\left( 2^{N_0/2}\prod_{i=1}^{J}\phi^{\mu_i}_{m_i}\right).
\end{equation}

From the unit-spin and one-half-spin wave functions, we can build a
wave function of arbitrary half-integer spin $J+1/2$:
\begin{equation}
  \Psi^{\mu_0\mu_1\dots\mu_J}_M = 
  \sum_{\vec{m}} \psi^{\mu_0}_{m_0}\prod_{i=1}^{J} \cg{i{-}\tfrac{1}{2}}{m_{0:i{-}1}}{1}{m_i}{i{+}\tfrac{1}{2}}{m_{0:i}}\, \phi^{\mu_i}_{m_i},
\end{equation}
again with the sum over $\vec m$ being over vectors that fulfill
$m_{0:J} = M$.  Since the order of the coupling does not matter, the
half-spin coupling can also be arranged last:
\begin{equation}
  \Psi^{\mu_0\mu_1\dots\mu_J}_M = \sum_{\lambda=\pm\frac{1}{2}} \cg{J}{m}{\tfrac{1}{2}}{\lambda}{J{+}\tfrac{1}{2}}{M}\, \psi^{\mu_0}_{\lambda}\, \Phi_{m}^{\mu_1\dots\mu_J}
\end{equation}
with
\begin{equation}
  \cg{J}{m}{\tfrac{1}{2}}{\pm\tfrac{1}{2}}{J{+}\tfrac{1}{2}}{m\pm\tfrac{1}{2}} = \left[\frac{J\pm m +1}{2J+1}\right]^{\frac{1}{2}}
\end{equation}

For example, the wave functions for spin $3/2$ are
\begin{equation}
  \Psi^{\alpha\mu}_{\pm\frac{3}{2}} = \psi^\alpha_\pm \phi^\mu_\pm \qquad
  \Psi^{\alpha\mu}_{\pm\frac{1}{2}} = \sqrt\frac{1}{3} \psi^\alpha_\mp\phi^\mu_\pm + \sqrt\frac{2}{3} \psi^\alpha_\pm\phi^\mu_0;
\end{equation}
and for spin $5/2$ are
\begin{align}
  \Psi^{\alpha\mu\nu}_{\pm\frac{5}{2}} & = \psi^\alpha_\pm \phi^\mu_\pm \phi^\nu_\pm \\
  \Psi^{\alpha\mu\nu}_{\pm\frac{3}{2}} & = \sqrt\frac{1}{5} \psi^\alpha_\mp \phi^\mu_\pm \phi^\nu_\pm + \sqrt\frac{2}{5} \psi^\alpha_\pm (\phi^\mu_\pm \phi^\nu_0 + \phi^\mu_0 \phi^\nu_\pm) \\
  \Psi^{\alpha\mu\nu}_{\pm\frac{1}{2}} & = \sqrt\frac{1}{5} \psi^\alpha_\mp  (\phi^\mu_\pm \phi^\nu_0 + \phi^\mu_0 \phi^\nu_\pm)
  + \sqrt\frac{1}{10} \psi^\alpha_\pm (\phi^\mu_\pm \phi^\nu_\mp + 2\phi^\mu_0 \phi^\nu_0 + \phi^\mu_\mp\phi^\nu_\pm)
\end{align}

\section{Two-body decay}

Let us describe the decay \decay{(0)}{(1)+(2)} of particles with spins
$j_{0}$, $j_{1}$, and $j_{2}$. The initial state is taken at
rest, so $\gamma_{0} = 1$ and $\beta_{0} = 0$. The final states,
$(1)$ and $(2)$ are back to back, so in the common coordinate
system---with $\hat{z}$ defined by the $(1)$ direction---the wave
functions for $(2)$ have a negative sign appearing with all factors of
$\beta_{2}$.

The wave functions for the states, generically given for integer and
half-integer spins are $\chi_{i}$ with spin projections $\lambda_{i}$.

The state of total intrinsic spin, $S$, of the final states, with spin
projection $m$, is
\begin{align}
  \Omega^{j_1j_2}_{sm} &\equiv \Omega_{sm}^{(\mu_0)\mu_1\dots\mu_{\lfloor j_1\rfloor}(\nu_0)\nu_1\dots\nu_{\lfloor j_2\rfloor}}\\
    & = \sum_{m_1, m_2} \cg{j_1}{m_1}{j_2}{m_2}{s}{m}
    \chi_{(1)m_1}^{(\mu_0)\mu_1\dots\mu_{\lfloor j_1\rfloor}}
    \chi_{(2)m_2}^{(\nu_0)\nu_1\dots\nu_{\lfloor j_2\rfloor}}
\end{align}

The orbital angular momentum, $\ell$, is represented by the wave
function
\begin{equation}
  \Phi_{\ell0} \equiv \Phi^{\mu_1\dots\mu_\ell}_0
\end{equation}

The decay amplitude has the form
\begin{equation}
  F^{\vec j}_{\vec m} = \sum_{\ell s} \cg{j_1}{m_1}{j_2}{{-}m_2}{s}{m_0} a^{\vec j}_{\ell s} A^{\vec j m_0}_{\ell s} r^\ell,
\end{equation}
where $a_{\ell s}$ is an empirical amplitude, $r$ is the breakup
momentum in the $(0)$ rest frame, and $A^{\vec j m_0}_{\ell s}$
covariant decacy amplitude for the decay
\decay{\ket{j_0m_0}}{\ket{sm_0} + \ket{\ell0}}; $\ket{sm_0}$ is the
coupling of $\ket{j_1m_1}$ and $\ket{j_2m_2}$ to total intrinsic spin
$s$ and spin projection $m_0 = m_1 - m_2$, and $\ket{\ell0}$ is the
state of total orbital angular momentum $\ell$.

\begin{equation}
  A^{\vec j m_0}_{\ell s} = p_0^n \otimes \Omega^{j_1j_2}_{sm_0} \otimes \Phi_{\ell 0} \otimes \chi_{(0)},
\end{equation}
with $n = (\sum j_i + \ell) \mod 2$

\bibliography{formalism}
\bibliographystyle{plain}

\end{document}
